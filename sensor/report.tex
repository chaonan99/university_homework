% !TEX encoding = UTF-8
\documentclass[a4paper,12pt]{article}
\usepackage{graphicx}
\usepackage{geometry}
\geometry{left=3.4cm,right=3.4cm,top=4.0cm,bottom=3.2cm}

\usepackage{booktabs}
\usepackage{array}
\usepackage{paralist}
\usepackage{verbatim}
\usepackage{subfig}
\usepackage{amsmath}
\usepackage{mathtools}
\usepackage{listings}
\usepackage[table]{xcolor}
\usepackage{lastpage}
\usepackage{url}

% Using hyperref for improved ref character
\usepackage[colorlinks,linkcolor=black,anchorcolor=black,
citecolor=black,CJKbookmarks=True]{hyperref}

% For picture drawing
\usepackage[all]{xy}

% For code inserting. Set features.
\lstset{
alsolanguage=matlab,
tabsize=4,
keepspaces=true,
numbers=left,
numberstyle=\tiny,
keywordstyle=\color{blue!70} \bfseries,
commentstyle=\color{red!50!green!50!blue!50},
frame=shadowbox,
breaklines,
showspaces=false,
showstringspaces=false,
showtabs=false,
rulesepcolor=\color{red!20!green!20!blue!20},
extendedchars=false,
escapeinside=``
}

% Set the font of page header
\usepackage{fancyhdr}
\pagestyle{fancy}
\lhead{Technical Report}
\chead{}
\rhead{Page \thepage/\pageref{LastPage}}
\cfoot{}
\rfoot{}
\lfoot{}

\usepackage{sectsty}

\usepackage[nottoc]{tocbibind}
\usepackage[titles,subfigure]{tocloft}
\renewcommand{\cftsecfont}{\rmfamily\mdseries\upshape}
\renewcommand{\cftsecpagefont}{\rmfamily\mdseries\upshape}

% Set number of ref to be relevent to section number
\renewcommand{\theequation}{\arabic{section}.\arabic{equation}}
\renewcommand{\thefigure}{\arabic{section}-\arabic{figure}}
\renewcommand{\thetable}{\arabic{section}-\arabic{table}}
\makeatletter
\@addtoreset{equation}{section}
\@addtoreset{figure}{section}
\@addtoreset{table}{section}
\makeatother

% Set the font of the reference
\bibliographystyle{unsrt}

% Define user\rq{}s color
\usepackage{colortbl}
\definecolor{lightgray}{gray}{.9}
\definecolor{thickgray}{gray}{.6}

\usepackage{multirow}

% Two spaces for new paragraph
\usepackage{indentfirst}

% Abstract
\usepackage{titling}
\usepackage{lipsum}

% % Set section numbering
% \CTEXsetup[number={}]{part}
% \renewcommand{\thepart}{}
% \usepackage{titlesec}
% \titleformat{\part}[block]{\color{blue}\huge\bfseries\filcenter}{}{1em}{}


\begin{document}
%%%%%%%%%%%%%%%%%%%%%%%%%%%%封面与目录%%%%%%%%%%%%%%%%%%%%%%%%%%%%%%
\begin{titlepage}
\begin{center}
% Upper part of the page
\includegraphics[width=0.25\textwidth]{resource/logo.jpg}\\[1cm]
\textsc{\LARGE Department of Automation}\\[1.5cm]
\fs{\Large 检测原理系列实验}\\[0.5cm]
% Title
\hrulefill
\\[0.8cm]{\centering \huge \hei 传感器特性实验报告}\\[0.4cm]
\hrulefill
\\[4cm]

% Author and supervisor
\begin{tabbing}       %tabbing  列表

 \hspace*{5cm} \= \hspace{2.6cm} \= \kill
 % \=     in tabbing environment, sets a tab stop
 % \kill  in a\tabbing environment, deletes previous line so tabs can be set without outputting text.
 % \>     in tabbing environment is a forward tab.

\>{\fs\sihao\textbf {班\hspace{1cm}级 \ \ :}}\>  {\centering\fs\sihao\textbf{~~~~~~~~~~~自~~3~2}} \\
\\
\>{\fs\sihao\textbf {姓\hspace{1cm}名 \ \ :}}\>  {\centering\fs\sihao\textbf{~陈~昊~楠~(2013011449)}}\\
\\
\>{\fs\sihao\textbf {同\hspace{0.2cm}组\hspace{0.3cm}人 \ \ :}}\>  {\centering\fs\sihao\textbf{~陈~炜~祥~(2013011456)}}\\
\\
\>{\fs\sihao\textbf {指导教师 \ \ :}}\>  {\centering\fs\sihao\textbf{~~~~~~~~~~~陆~~~~~耿}} \\

\end{tabbing}
\vfill
{\large \today}
\end{center}
\end{titlepage}

\tableofcontents
\clearpage

%%%%%%%%%%%%%%%%%%%%%%%%%%正文部分%%%%%%%%%%%%%%%%%%%%%%%%%%%%%%%%%%


\begin{figure}[htbp]
\centering
\includegraphics[width=11cm]{resource/principle.png}
\caption{电感式传感器原理框图}
\label{fig:sp}
\end{figure}

5.实验报告要求
(1) 简述电感式传感器的工作原理。
(2) 记录实验数据并作必要的分析与计算。
(3) 回答思考题。

\part{电感式传感器特性研究}
\section{工作原理}
电感式传器由振荡、开关路和放大输出三部分组成,如图\ref{fig:sp}所示。接通电源,振荡器开始振荡并产生一个交变的磁场,当金属目标板接近这一磁场并达到感应距离时,在目标板内产生涡流,从而导致振荡衰减,以至停振;振荡器从振荡到停振的变化被后级放大电路处理并转换成开关信号,从而达到检测目标板的目的。该传感器只能检测金属目标,而不同金属材料的工作点距离不同。
\begin{figure}[htbp]
\centering
\includegraphics[width=11cm]{resource/principle.png}
\caption{电感式传感器原理框图}
\label{fig:sp}
\end{figure}

\section{实验内容与数据}
\paragraph{电感式传感器的感应范围和回差}
\begin{table}
	\centering
	\begin{tabular}{|c|c|c|}
		\hline
		工作点 & 释放点 & 回差 \\
		\hline
		PR\_DATA02.txt & dataset1.txt & 9.8837\% & 12.5000\% \\
		dataset1.txt & PR\_DATA02.txt & 8.8415\% & 11.6279\% \\
		\hline
	\end{tabular}
	\caption{电感式传感器的感应范围和回差}
	\label{tab:srr}
\end{table}
\section{思考题}

\part{磁式传感器特性研究}
\section{工作原理}
\section{实验内容与数据}
\section{思考题}
\part{光电传感器特性研究}
\section{工作原理}
\section{实验内容与数据}
\section{思考题}
\part{超声波传感器特性研究实验}
\section{工作原理}
\section{实验内容与数据}
\section{思考题}
\part{模拟量传感器特性研究(电感式)}
\section{工作原理}
\section{实验内容与数据}
\section{思考题}

\end{document}
