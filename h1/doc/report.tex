% !TEX encoding = UTF-8
\documentclass[a4paper,12pt]{article}
\usepackage{graphicx}
\usepackage{geometry}
\geometry{left=3.4cm,right=3.4cm,top=4.0cm,bottom=3.2cm}

\usepackage{booktabs}
\usepackage{array}
\usepackage{paralist}
\usepackage{verbatim}
\usepackage{subfig}
\usepackage{amsmath}
\usepackage{mathtools}
\usepackage{listings}
\usepackage[table]{xcolor}
\usepackage{lastpage}
\usepackage{url}

% Using hyperref for improved ref character
\usepackage[colorlinks,linkcolor=black,anchorcolor=black,
citecolor=black,CJKbookmarks=True]{hyperref}

% For picture drawing
\usepackage[all]{xy}

% For code inserting. Set features.
\lstset{
alsolanguage=matlab,
tabsize=4,
keepspaces=true,
numbers=left,
numberstyle=\tiny,
keywordstyle=\color{blue!70} \bfseries,
commentstyle=\color{red!50!green!50!blue!50},
frame=shadowbox,
breaklines,
showspaces=false,
showstringspaces=false,
showtabs=false,
rulesepcolor=\color{red!20!green!20!blue!20},
extendedchars=false,
escapeinside=``
}

% Set the font of page header
\usepackage{fancyhdr}
\pagestyle{fancy}
\lhead{Technical Report}
\chead{}
\rhead{Page \thepage/\pageref{LastPage}}
\cfoot{}
\rfoot{}
\lfoot{}

\usepackage{sectsty}

\usepackage[nottoc]{tocbibind}
\usepackage[titles,subfigure]{tocloft}
\renewcommand{\cftsecfont}{\rmfamily\mdseries\upshape}
\renewcommand{\cftsecpagefont}{\rmfamily\mdseries\upshape}

% Set number of ref to be relevent to section number
\renewcommand{\theequation}{\arabic{section}.\arabic{equation}}
\renewcommand{\thefigure}{\arabic{section}-\arabic{figure}}
\renewcommand{\thetable}{\arabic{section}-\arabic{table}}
\makeatletter
\@addtoreset{equation}{section}
\@addtoreset{figure}{section}
\@addtoreset{table}{section}
\makeatother

% Set the font of the reference
\bibliographystyle{unsrt}

% Define user\rq{}s color
\usepackage{colortbl}
\definecolor{lightgray}{gray}{.9}
\definecolor{thickgray}{gray}{.6}

\usepackage{multirow}

% Two spaces for new paragraph
\usepackage{indentfirst}

% Abstract
\usepackage{titling}
\usepackage{lipsum}

% % Set section numbering
% \CTEXsetup[number={}]{part}
% \renewcommand{\thepart}{}
% \usepackage{titlesec}
% \titleformat{\part}[block]{\color{blue}\huge\bfseries\filcenter}{}{1em}{}


\begin{document}
%%%%%%%%%%%%%%%%%%%%%%%%%%%%封面与目录%%%%%%%%%%%%%%%%%%%%%%%%%%%%%%
\begin{titlepage}
\begin{center}
% Upper part of the page
\includegraphics[width=0.25\textwidth]{./logo.jpg}\\[1cm]
\textsc{\LARGE Department of Automation}\\[1.5cm]
\fs{\Large 模式识别基础第一次作业}\\[0.5cm]
% Title
\hrulefill
\\[0.8cm]{\centering \huge \hei 用身高体重数据进行性别分类的实验(一)}\\[0.4cm]
\hrulefill
\\[4cm]

% Author and supervisor
\begin{tabbing}       %tabbing  列表

 \hspace*{5cm} \= \hspace{2.6cm} \= \kill
 % \=     in tabbing environment, sets a tab stop
 % \kill  in a\tabbing environment, deletes previous line so tabs can be set without outputting text.
 % \>     in tabbing environment is a forward tab.

\>{\fs\sihao\textbf {班\hspace{1cm}级 \ \ :}}\>  {\centering\fs\sihao\textbf{~~~~~~~~~自~~3~2}} \\
\\
\>{\fs\sihao\textbf {姓\hspace{1cm}名 \ \ :}}\>  {\centering\fs\sihao\textbf{~~~~~~~~陈~昊~楠}}\\
\\
\>{\fs\sihao\textbf {学\hspace{1cm}号 \ \ :}}\>  {\centering\fs\sihao\textbf{~~~~~~2013011449}}\\
\\
\>{\fs\sihao\textbf {授课教师 \ \ :}}\>  {\centering\fs\sihao\textbf{~~~~~~~~张~学~工}} \\

\end{tabbing}
\vfill
{\large \today}
\end{center}
\end{titlepage}

\tableofcontents
\clearpage

%%%%%%%%%%%%%%%%%%%%%%%%%%正文部分%%%%%%%%%%%%%%%%%%%%%%%%%%%%%%%%%%

\section{实验内容}
收集至少50 个人的性别、身高、体重等数据,结合给定的数据集,采用身高和体重为特征进行性别分类。根据贝叶斯决策原理,使用最小错误率贝叶斯分类器进行二分类任务。具体步骤如下:

\begin{itemize}
	\item 在正态分布假设下估计概率密度,自行确定先验概率,建立最小错误率贝叶斯分类器,写出得到的决策规则;计算训练错误率和在 dataset1.txt 上的测试错误率;
	\item 自行给出一个决策表,采用最小风险贝叶斯决策重复上面的实验;
	\item 用题中得到的似然比或对数似然比为分类指标,粗略画出ROC 曲线,体会两类错误率的概念和与最小风险决策的关系;
	\item 交换训练集和测试集,重复上述步骤 1~3。
\end{itemize}

\section{实验方法}


\section{数据说明}
	\paragraph{采集方式} 使用线上问卷平台“问卷星”发放问卷。问卷链接:\url{https://www.sojump.hk/jq/9817519.aspx}。
	\paragraph{小组成员名单} 安亮/\textbf{陈昊楠}/陈炜祥/冯乔俊/刘浪/陆欣然/牛家禾/欧阳佳弘/孙世杰/万琳莉/王柏然/王梓言/吴迪/尹彦歆
	\paragraph{采集结果} 总共收集数据356 份,其中有效数据344 份,无效数据12 份。该数据经过\textbf{冯乔俊}同学的预处理,处理方法为:
	\begin{itemize}
		\item 女性,填写体重值80 以上,体重值统一除以2;
		\item 男性,填写体重值100 以上,体重值统一除以2;
		\item 男性,填写体重值100 以上,且身高值170以下,体重值统一除以2。
	\end{itemize}
	由此得到有效数据344 份。女性F 共172 份,男性M 共172 份。

\section{实验结果}
\subsection{最小错误率决策}
	\paragraph{先验概率估计} 此处应该估计实际情况中男生、女生的比例。由于不知道测试数据的来源,此处认为男生女生各占一半,即两类的先验概率为:
	\begin{equation}
		P(\omega_1)=P(\omega_2)=1/2
	\end{equation}
	\paragraph{概率密度估计} 假设身高、体重数据服从二维正态分布,从数据分布情况看,不能假设身高体重独立,亦不能假设男、女的身高体重分布具有相同的协方差矩阵,因此分别当做任意的正态分布估计参数。采集数据的估计结果为
	\begin{align}
		\mu_1 &= [175.9297,68.5314]\\
		\Sigma_1 &= \begin{bmatrix}
		32.02767 & 25.56843 \\
		25.56843 & 101.24157
		\end{bmatrix}\\
		\mu_2 &= [163.54651,52.42209]\\
		\Sigma_2 &= \begin{bmatrix}
		26.42226 & 12.93734 \\
		12.93734 & 30.61347
		\end{bmatrix}
	\end{align}
	提供数据的估计结果为
	\begin{align}
		\mu_1 &= [174.996,67.234]\\
		\Sigma_1 &= \begin{bmatrix}
		28.31398 & 22.64894 \\
		22.64894 & 97.98624
		\end{bmatrix}\\
		\mu_2 &= [162.32051,51.40385]\\
		\Sigma_2 &= \begin{bmatrix}
		20.910092 & 8.300049 \\
		8.300049 & 26.776011
		\end{bmatrix}
	\end{align}
	\paragraph{判别函数与决策规则} 在多维正态分布假设下,采用对数形式判别函数:
	\begin{equation}
		g_i(\mathbf{x})=-\frac{1}{2}(\mathbf{x}-\mu_i)^\mathbf{T}\Sigma^{-1}_i(\mathbf{x}-\mu_i)-\frac{d}{2}\ln 2\pi-\frac{1}{2}\ln|\Sigma_i|+\ln P(\omega_i)
	\end{equation}
	其中$i=1,2$分别表示男、女。将身高、体重向量带入两个判别函数,决策规则为
	\begin{equation}
		\text{若} \quad g_1(\mathbf{x})-g_2(\mathbf{x}) \gtrless 0 \text{,则}\quad
		\mathbf{x} \in \begin{dcases}
		\omega_1\\
		\omega_2
		\end{dcases}
	\end{equation}
	\paragraph{错误率} 表\ref{tab:minerror}为分别将采集数据、提供的数据作为训练、测试集并互换后错误率的结果。
	\begin{table}
		\centering
		\begin{tabular}{|c|c|c|c|}
			\hline
			训练集 & 测试集 & 训练集错误率 & 测试集错误率 \\
			\hline
			PR\_DATA02.txt & dataset1.txt & 9.8837\% & 12.5000\% \\
			dataset1.txt & PR\_DATA02.txt & 6.4024\% & 13.9535\% \\
			\hline
		\end{tabular}
		\caption{最小错误率决策结果}
		\label{tab:minerror}
	\end{table}

\subsection{最小风险决策}
\paragraph{决策表} 设决策与状态相同时损失为0,此时仅$\lambda_{12}$与$\lambda_{21}$的比值影响决策结果。因此固定$\lambda_{21}=1$,改变$\lambda_{12}$,即采用表\ref{tab:lossform}所示的决策表。
\begin{table}
	\centering
	\begin{tabular}{c|cc}
	\hline
	& $\omega_1$ & $\omega_2$ \\
	\hline
	$\alpha_1$ & 0 & $\lambda_{12}$\\
	$\alpha_2$ & 1 & 0 \\
	\hline
	\end{tabular}
	\label{tab:lossform}
	\caption{决策表}
\end{table}
\paragraph{判别函数与决策规则} 仍然采用上一节的判别函数,此时决策规则变为
\begin{equation}
	\text{若} \quad g_1(\mathbf{x})-g_2(\mathbf{x}) \gtrless \ln\lambda_{12} \text{,则}\quad
	\mathbf{x} \in \begin{dcases}
	\omega_1\\
	\omega_2
	\end{dcases}
\end{equation}
\paragraph{错误率} 表\ref{tab:minrisk}为分别将采集数据、提供的数据作为训练、测试集并互换后错误率的结果。
\begin{table}
	\centering
	\begin{tabular}{|c|c|c|c|c|}
		\hline
		& \multicolumn{2}{|c|}{采集数据作为训练集} & \multicolumn{2}{|c|}{采集数据作为测试集} \\
		\hline
		$\lambda_{12}$ & 训练集错误率 & 测试集错误率 & 训练集错误率 & 测试集错误率 \\
		\hline
		0.001 & & & & \\
		0.003 & & & & \\
		0.01 & & & & \\
		0.03 & & & & \\
		0.1 & & & & \\
		0.3 & & & & \\
		1 & & & & \\
		3 & & & & \\
		10 & & & & \\
		30 & & & & \\
		\hline
	\end{tabular}
	\caption{最小风险决策结果}
	\label{tab:minrisk}
\end{table}
\section{分析与体会}

% \begin{figure}[htbp]
% \centering
% \includegraphics[width=11cm]{logo.jpg}
% \caption{全桥测试电路}
% \label{fig:fullarm}
% \end{figure}

\end{document}
