% !TEX encoding = UTF-8
\documentclass[a4paper,12pt]{article}
\usepackage{graphicx}
\usepackage{geometry}
\geometry{left=3.4cm,right=3.4cm,top=4.0cm,bottom=3.2cm}

\usepackage{booktabs}
\usepackage{array}
\usepackage{paralist}
\usepackage{verbatim}
\usepackage{subfig}
\usepackage{amsmath}
\usepackage{mathtools}
\usepackage{listings}
\usepackage[table]{xcolor}
\usepackage{lastpage}
\usepackage{url}

% Using hyperref for improved ref character
\usepackage[colorlinks,linkcolor=black,anchorcolor=black,
citecolor=black,CJKbookmarks=True]{hyperref}

% For picture drawing
\usepackage[all]{xy}

% For code inserting. Set features.
\lstset{
alsolanguage=matlab,
tabsize=4,
keepspaces=true,
numbers=left,
numberstyle=\tiny,
keywordstyle=\color{blue!70} \bfseries,
commentstyle=\color{red!50!green!50!blue!50},
frame=shadowbox,
breaklines,
showspaces=false,
showstringspaces=false,
showtabs=false,
rulesepcolor=\color{red!20!green!20!blue!20},
extendedchars=false,
escapeinside=``
}

% Set the font of page header
\usepackage{fancyhdr}
\pagestyle{fancy}
\lhead{Technical Report}
\chead{}
\rhead{Page \thepage/\pageref{LastPage}}
\cfoot{}
\rfoot{}
\lfoot{}

\usepackage{sectsty}

\usepackage[nottoc]{tocbibind}
\usepackage[titles,subfigure]{tocloft}
\renewcommand{\cftsecfont}{\rmfamily\mdseries\upshape}
\renewcommand{\cftsecpagefont}{\rmfamily\mdseries\upshape}

% Set number of ref to be relevent to section number
\renewcommand{\theequation}{\arabic{section}.\arabic{equation}}
\renewcommand{\thefigure}{\arabic{section}-\arabic{figure}}
\renewcommand{\thetable}{\arabic{section}-\arabic{table}}
\makeatletter
\@addtoreset{equation}{section}
\@addtoreset{figure}{section}
\@addtoreset{table}{section}
\makeatother

% Set the font of the reference
\bibliographystyle{unsrt}

% Define user\rq{}s color
\usepackage{colortbl}
\definecolor{lightgray}{gray}{.9}
\definecolor{thickgray}{gray}{.6}

\usepackage{multirow}

% Two spaces for new paragraph
\usepackage{indentfirst}

% Abstract
\usepackage{titling}
\usepackage{lipsum}

% % Set section numbering
% \CTEXsetup[number={}]{part}
% \renewcommand{\thepart}{}
% \usepackage{titlesec}
% \titleformat{\part}[block]{\color{blue}\huge\bfseries\filcenter}{}{1em}{}


\begin{document}
%%%%%%%%%%%%%%%%%%%%%%%%%%%%封面与目录%%%%%%%%%%%%%%%%%%%%%%%%%%%%%%
\begin{titlepage}
\begin{center}
% Upper part of the page
\includegraphics[width=0.25\textwidth]{resource/logo.jpg}\\[1cm]
\textsc{\LARGE Department of Automation}\\[1.5cm]
\fs{\Large 模式识别基础第二次作业}\\[0.5cm]
% Title
\hrulefill
\\[0.8cm]{\centering \huge \hei 用身高体重数据进行性别分类的实验(二)}\\[0.4cm]
\hrulefill
\\[4cm]

% Author and supervisor
\begin{tabbing}       %tabbing  列表

 \hspace*{5cm} \= \hspace{2.6cm} \= \kill
 % \=     in tabbing environment, sets a tab stop
 % \kill  in a\tabbing environment, deletes previous line so tabs can be set without outputting text.
 % \>     in tabbing environment is a forward tab.

\>{\fs\sihao\textbf {班\hspace{1cm}级 \ \ :}}\>  {\centering\fs\sihao\textbf{~~~~~~~~~自~~3~2}} \\
\\
\>{\fs\sihao\textbf {姓\hspace{1cm}名 \ \ :}}\>  {\centering\fs\sihao\textbf{~~~~~~~~陈~昊~楠}}\\
\\
\>{\fs\sihao\textbf {学\hspace{1cm}号 \ \ :}}\>  {\centering\fs\sihao\textbf{~~~~~~2013011449}}\\
\\
\>{\fs\sihao\textbf {授课教师 \ \ :}}\>  {\centering\fs\sihao\textbf{~~~~~~~~张~学~工}} \\

\end{tabbing}
\vfill
{\large \today}
\end{center}
\end{titlepage}

\tableofcontents
\clearpage

%%%%%%%%%%%%%%%%%%%%%%%%%%正文部分%%%%%%%%%%%%%%%%%%%%%%%%%%%%%%%%%%

\section{实验内容}
\begin{enumerate}
	\item 用 {\ttfamily dataset3} 作为训练数据,用 {\ttfamily dataset4} 作为测试数据,采用不同的特征、训练样本数、分类方法进行比较实验,观察、分析实验结果的异同。
	\begin{itemize}
		\item 特征组合
		\begin{enumerate}
			\item 10个特征都用
			\item 只用第3、5列特征
		\end{enumerate}
		\item 训练样本组合
		\begin{enumerate}
			\item 从dataset3 中任选20 个训练样本(男女各10 例)
			\item dataset3 中的全部训练样本
		\end{enumerate}
		\item 分类器方法
		\begin{enumerate}
			\item 最小错误率贝叶斯分类器(假设正态分布,先验概率各50\%)
			\item Fisher 线性判别(FLD)
			\item 线性SVM
			\item 采用BP 算法的MLP 神经网络(网络结构自定)
		\end{enumerate}
	\end{itemize}
	\item 汇总测试错误率
	\item 结合实验观察和对各种方法特点的理解,尝试对训练样本数、特征维数以及所选用的方法与测试结果的关系进行讨论。
	\item (选做)选择具有代表性的实验结果,设法在由第3 列和第5 列特征组成的平面上粗略画出不同样本数和特征数下得到的分类器,观察分析所得分类器特性与样本数、特征、分类方法之间的关系。
\end{enumerate}

\section{实验方法}
\paragraph{Bayes最小错误率分类}
该方法假设男性、女性两类之内对所有特征服从多维正态分布。设两类的均值、协方差矩阵分别为$\mu_i,\Sigma_i\quad i=1,2$,则决策边界可按式\autoref{eq:bayes}计算:
\begin{equation}
\label{eq:bayes}
-\frac12[(x-\mu_i)^\mathbf{T}\Sigma_i^{-1}(x-\mu_i)-(x-\mu_j)^\mathbf{T}\Sigma_j^{-1}(x-\mu_j)]-\frac12\ln \frac{|\Sigma_i|}{|\Sigma_j|}+\ln\frac{P(\omega_i)}{P(\omega_j)}=0
\end{equation}
若假设先验概率均为$0.5$,则上式最后一项可以忽略。
\paragraph{Fisher线性分类器}
二分类的线性判别函数的一般表达式为
\begin{equation}
f(\mathbf{x}) = \mathbf{w}^T \mathbf{x} + b
\end{equation}
该方法可以看作将所有样本投影到一维,并根据阈值$b$进行分类。Fisher线性判别的思想就是,选择投影方向,使得投影后两类相隔尽可能远,而同时每一类内部的样本又尽可能聚集。其数学表达为
\begin{equation}
\max J_F(w)=\frac{\tilde{S}_b}{\tilde{S}_w}=\frac{(\tilde{m}_1-\tilde{m}_2)^2}{\tilde{S}_1^2+\tilde{S}_2^2}
\end{equation}
其中$\tilde{m}_1, \tilde{m}_2$为两类投影后的均值
\begin{equation}
\tilde{m}_i=\frac1{N_i}\sum_{y_j\in \mathcal{Y}_i}y_j=\frac1{N_i}\sum_{x_j\in \mathcal{X}_i}w^\mathbf{T}\boldsymbol{x_j}=w^\mathbf{T}\boldsymbol{\mu}_i\quad i=1,2
\end{equation}
$\tilde{S}_1, \tilde{S}_2$为两类投影后的的类内离散度
\begin{equation}
\tilde{S}_i^2=\sum_{y_j\in \mathcal{Y}_i}(y_j-\tilde{m}_i)^2\quad i=1,2
\end{equation}
可以证明,在该准则下,投影方向$w$为
\begin{equation}
\mathbf{w} = \alpha \cdot (\mathbf{S}_1 + \mathbf{S}_2)^{-1} (\boldsymbol{\mu}_1 - \boldsymbol{\mu}_2)
\end{equation}
阈值为
\begin{equation}
b = -\frac12(\boldsymbol{\mu}_1 + \boldsymbol{\mu}_2)(\mathbf{S}_1 + \mathbf{S}_2)^{-1} (\boldsymbol{\mu}_1 - \boldsymbol{\mu}_2)-\ln\frac{P(\omega_1)}{P(\omega_2)}
\end{equation}
其中$\mathbf{S}_1, \mathbf{S}_2$为类内离散度矩阵
\begin{equation}
\mathbf{S}_i=\sum_{x_j\in \mathcal{X}_i}(\boldsymbol{x_j}-\boldsymbol{\mu}_i)(\boldsymbol{x_j}-\boldsymbol{\mu}_i)^\mathbf{T}
\end{equation}

\section{实验结果}
错误率汇总如表\ref{tab:error}

\begin{table}
	\centering
	\begin{tabular}{c|c|cccc}
	\hline
	训练样本数 & 特征数 & Bayes & FLD & Linear SVM & MLP (hidden nodes=12) \\
	\hline
	\multirow{2}*{10+10} & 10 & 12.538226\% & 26.299694\% & 24.464832\% & 72.47706422\% (epoch=100)\\
		\cline{2-6}
		& 2 & 22.324159\% & 15.596330\% & 17.125382\% & 82.87461774\% (epoch=100) \\
	\hline
	\multirow{2}*{469+485} & 10 & 16.819572\% & 10.091743\% & 11.926606\% & 11.62079511\% (epoch=500) \\
		\cline{2-6}
		& 2 & 	14.067278\% & 11.620795\% & 12.232416\% & 7.339449541\% (epoch=500) \\
	\hline
	\end{tabular}
	\label{tab:error}
	\caption{各种方法错误率汇总}
\end{table}

\section{分析与体会}


\section{数据与程序说明}
\subsection{随机采样方法}
采用{\ttfamily sample}函数生成一定范围内不重复的随机整数。例如:
\begin{lstlisting}
	sample(1:size(df_train, 1), 10, replace = false)
\end{lstlisting}
函数具体说明参见 \url{http://statsbasejl.readthedocs.io/en/latest/sampling.html}。挑选出的样本参见源程序 \url{http://nbviewer.jupyter.org/github/chaonan99/PR_hw/blob/master/h2/src/h2.ipynb}。

\subsection{程序说明}
程序采用{\ttfamily Julia}语言,由{\ttfamily IJulia}提供对{\ttfamily Jupyter Notebook}的支持。
\paragraph{简要方法描述} 见实验方法及结果部分。
\paragraph{程序来源} 源代码均为本人编写。用到的包参见源程序中的说明。
\paragraph{参数说明} 参见源程序中的说明。

\begin{figure}[htbp]
\centering
\includegraphics[width=11cm]{resource/logo.jpg}
\caption{全桥测试电路}
\label{fig:fullarm}
\end{figure}

\end{document}
