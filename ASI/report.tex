% !TEX encoding = UTF-8
\documentclass[a4paper,12pt]{article}
\usepackage{graphicx}
\usepackage{geometry}
\geometry{left=3.4cm,right=3.4cm,top=4.0cm,bottom=3.2cm}

\usepackage{booktabs}
\usepackage{array}
\usepackage{paralist}
\usepackage{verbatim}
\usepackage{subfig}
\usepackage{amsmath}
\usepackage{mathtools}
\usepackage{listings}
\usepackage[table]{xcolor}
\usepackage{lastpage}
\usepackage{url}

% Using hyperref for improved ref character
\usepackage[colorlinks,linkcolor=black,anchorcolor=black,
citecolor=black,CJKbookmarks=True]{hyperref}

% For picture drawing
\usepackage[all]{xy}

% For code inserting. Set features.
\lstset{
alsolanguage=matlab,
tabsize=4,
keepspaces=true,
numbers=left,
numberstyle=\tiny,
keywordstyle=\color{blue!70} \bfseries,
commentstyle=\color{red!50!green!50!blue!50},
frame=shadowbox,
breaklines,
showspaces=false,
showstringspaces=false,
showtabs=false,
rulesepcolor=\color{red!20!green!20!blue!20},
extendedchars=false,
escapeinside=``
}

% Set the font of page header
\usepackage{fancyhdr}
\pagestyle{fancy}
\lhead{Technical Report}
\chead{}
\rhead{Page \thepage/\pageref{LastPage}}
\cfoot{}
\rfoot{}
\lfoot{}

\usepackage{sectsty}

\usepackage[nottoc]{tocbibind}
\usepackage[titles,subfigure]{tocloft}
\renewcommand{\cftsecfont}{\rmfamily\mdseries\upshape}
\renewcommand{\cftsecpagefont}{\rmfamily\mdseries\upshape}

% Set number of ref to be relevent to section number
\renewcommand{\theequation}{\arabic{section}.\arabic{equation}}
\renewcommand{\thefigure}{\arabic{section}-\arabic{figure}}
\renewcommand{\thetable}{\arabic{section}-\arabic{table}}
\makeatletter
\@addtoreset{equation}{section}
\@addtoreset{figure}{section}
\@addtoreset{table}{section}
\makeatother

% Set the font of the reference
\bibliographystyle{unsrt}

% Define user\rq{}s color
\usepackage{colortbl}
\definecolor{lightgray}{gray}{.9}
\definecolor{thickgray}{gray}{.6}

\usepackage{multirow}

% Two spaces for new paragraph
\usepackage{indentfirst}

% Abstract
\usepackage{titling}
\usepackage{lipsum}

% % Set section numbering
% \CTEXsetup[number={}]{part}
% \renewcommand{\thepart}{}
% \usepackage{titlesec}
% \titleformat{\part}[block]{\color{blue}\huge\bfseries\filcenter}{}{1em}{}


\begin{document}
%%%%%%%%%%%%%%%%%%%%%%%%%%%%封面与目录%%%%%%%%%%%%%%%%%%%%%%%%%%%%%%
\begin{titlepage}
\begin{center}
% Upper part of the page
\includegraphics[width=0.25\textwidth]{resource/logo.jpg}\\[1cm]
\textsc{\LARGE Department of Automation}\\[1.5cm]
\fs{\Large 检测原理系列实验}\\[0.5cm]
% Title
\hrulefill
\\[0.8cm]{\centering \huge \hei 传感器特性实验报告}\\[0.4cm]
\hrulefill
\\[4cm]

% Author and supervisor
\begin{tabbing}       %tabbing  列表

 \hspace*{5cm} \= \hspace{2.6cm} \= \kill
 % \=     in tabbing environment, sets a tab stop
 % \kill  in a\tabbing environment, deletes previous line so tabs can be set without outputting text.
 % \>     in tabbing environment is a forward tab.

\>{\fs\sihao\textbf {班\hspace{1cm}级 \ \ :}}\>  {\centering\fs\sihao\textbf{~~~~~~~~~~~自~~3~2}} \\
\\
\>{\fs\sihao\textbf {姓\hspace{1cm}名 \ \ :}}\>  {\centering\fs\sihao\textbf{~陈~昊~楠~(2013011449)}}\\
\\
\>{\fs\sihao\textbf {同\hspace{0.2cm}组\hspace{0.3cm}人 \ \ :}}\>  {\centering\fs\sihao\textbf{~陈~炜~祥~(2013011456)}}\\
\\
\>{\fs\sihao\textbf {指导教师 \ \ :}}\>  {\centering\fs\sihao\textbf{~~~~~~~~~~~陆~~~~~耿}} \\

\end{tabbing}
\vfill
{\large \today}
\end{center}
\end{titlepage}

\tableofcontents
\clearpage

%%%%%%%%%%%%%%%%%%%%%%%%%%正文部分%%%%%%%%%%%%%%%%%%%%%%%%%%%%%%%%%%
4.实验报告要求
(1) 简述AS-i总线的原理和性能特点。
(2) AS-i专用电源有哪些特点?
(3) 记录实验中出现的现象并分析其工作原理。
(4) 回答思考题。
\part{AS-i总线技术(V3.0)特性研究}
\section{AS-i总线原理与特点}
\subsection{AS-i总线概述}
AS-i是“执行器-传感器-接口”的英文缩写(Actuator Sensor- interface)。它是一种用来在控制器(主站)和传感器/执行器(从站)之间双向交换信息、主从结构的总线网络,它属于现场总线下面底层的监控网络系统。\\
AS-i主站是AS-i总线系统的核心,它由AS-i主机和控制器(如PC和PLC等)组成。向上通过主站中的网关可以和多种现场总线(如PROFIBUS,FF,CANBUS等)进行连接,向下可以挂接一批AS-i从站,主站将按照AS-i通信协议与各个从站之间进行数据交换。\\
AS-i从站一般可分为两种类型,一种是智能型开关装置,它本身就带有从机专用芯片和配套电路,形成一体化从站,这种智能型传感器/执行器可以直接和AS-i总线连接。第二种是专门设计的AS-i 连接模块,在这种模块中带有从机专用芯片和配套电路,它除了具有通信接口外还带有I/O接口,这些I/O接口可以和普通的传感器/执行器相连接构成分离型从站。\\
AS-i系统主站和从站之间的通信采用非屏蔽、非绞接的两芯电缆。其中一种是普通的圆柱形电缆,另一种是专用的扁平电缆,由于扁平电缆采用一种特殊的穿刺安装方式把线压在连接件上,所以安装拆卸即简单又可靠。在两芯电缆上除传输信号外还通过网络向主站和从站提供电源。
\subsection{AS-i技术规范V3.0特点}
2006年AS-i国际协会发布了最新的AS-i技术规范V3.0,它比原有的V2.1版本有了更大的改进,并完全兼容以前版本的技术内容,最大的改进有以下两点:
\begin{enumerate}
\item 增加了AS-i安全系统元件的规范。
\item 增加了新的AS-i主站和从站规范以及组合处理类型(CTT,Combined Transaction Type)的定义。
\end{enumerate}
在自动化生产现场,不同的客户有不同的需求,特别是对于从站的性能、连接、安装都有不同的要求。在新的V3.0规范中,增加了很多新的从站规范,这也使生产商可以给客户提供更加多样化的产品。随着自动化产品在监控系统中的广泛应用,安全概念的不断变革和安全技术的不断更新,作为现场最底层的AS-i总线系统也在新的V3.0规范中增加了相关安全系统产品的规范。组合处理类型(CTT)的定义为AS-i系统的通讯增加了新的功能,包括对模拟量模块和安全输入/输出模块的通讯支持。
\section{实验内容与分析}
\section{思考题}
\part{实现PLC控制器对其外围I/O设备的控制}

\begin{figure}[htbp]
\centering
\includegraphics[width=11cm]{resource/logo.jpg}
\caption{全桥测试电路}
\label{fig:fullarm}
\end{figure}

\end{document}
